% arara: pdflatex
% arara: pdflatex

\documentclass{article}
\usepackage[utf8]{inputenc}
\usepackage[czech]{babel}
\usepackage{amssymb}
\usepackage{pgfplots}
\usepackage{float}
\usepackage{hyperref}
\usepackage{todonotes}


\title{Řešení problému vážené splnitelnosti booleovské formule pokročilou iterativní metodou}
\author{Tomáš Starý (staryto5)}

\begin{document}
    \maketitle
    \tableofcontents
    \pagebreak

    \section{Zadání úlohy}

    Podrobné zadání úlohy je na \href{https://moodle-vyuka.cvut.cz/mod/assign/view.php?id=48355}{této adrese}. Jedná se o řešení 3SAT problému pomocí některé z pokročlých heuristik:

    \begin{itemize}
        \item {simulované ochlazování}
        \item {genetický algoritmus}
        \item {tabu prohledávání}
    \end{itemize}

    Po nasazení heuristiky je potřeba ověřit její vhodnost a to pomocí experimentálního vyhodnocení.

    \subsection{Popis řešeného problému}

    Dána booleovská formule $F$ o $n$ proměnných, $X = (x_1, x_2,\dots,x_n)$ v konjunktivní normální formě (CNF).
    Dále jsou dány celočíselné kladné váhy těchto $n$ proměnných $W = (w_1, w_2,\dots, w_n)$. Nalezněte přiřazení $Y=(y_1, y_2,\dots, y_n)$
    proměnných $x_1, x_2,\dots, x_n$, takové, že $F(Y) = 1$ a součet $S$ vah proměnných nastavených do 1 je maximální.

    Omezte se na vážený 3-SAT problém, kde každá klauzule má právě 3 proměnné. Složitost problému je stejná, implementace a klasifikace instancí jsou jednodušší.

    \section{Zpracování a implementace řešení}

    Jako nejdříve sem začal parserem na zpracování souboru, u kterého sem kontroloval hodnoty, které ze souboru získal. Následovalo načtení dat do konečné struktury.

    Posléze jsem implementoval základní genetický algoritmus s křížením, mutací a výběrem jedinců do další generace pomocí turnajového výběru. Na menších instancích sem
    zkoušel chování programu a upravoval jednotlivé části, abych dosáhl optimálního výsledku.

    \todo{Dokončit}

    \subsection{Struktura programu}

    Program je psaný v jazyce C++, kompilován s maximální optimalizací (přepínač -O3) pomocí clang kompilátoru. Pro reprezentaci klauzulí a celého SAT problému byly
    použity třídy (Clause, SAT). Parser je poté samostatná funkce, které sestavuje třídu SATSolver, která vytváří ostatní objekty. Pro účely genetického algoritmu
    jsem implementoval ještě třídu Specimen, která reprezentuje jednotlivé členy v populaci.

    SATSolver je třída, která má na starosti vše okolo inicializace a řešení SAT problému. Nese v sobě informace o populaci, instanci SAT problému, velikosti populace,
    počtu klauzulí a počtu proměnných. Implementuje tyto metody:

    \begin{itemize}
        \item {SATSolver: Konstruktor třídy, jako vstupní data má pole vah jednotlivých proměnných a pole jednotlivých klauzulí. Na základě těchto dat vytvoří potřebnou
                datovou strukturu. Také vytáří základní populaci.}
        \item {Solve: Tato metoda slouží k spuštění a provedení samotné evoluce, která je vlastním řešením problému.}
        \item {Crossbreeding: Privátní metoda, má na starost samotné křížení mezi jednotlivými členy populace}
        \item {Mutation: Také privítní metoda, řeši mutaci jednotlivých členů populace.}
    \end{itemize}

    Třída SAT obsahuje všechny důležité informace o SAT problému. Nese informace o váhách proměnných a klauzulích, které dostane při zakládaní od SATSolver. Jednotlivé instance
    třídy Clause si ovšem třída zakládá sama z dat, které jsou jí předány v konstruktoru. Obsahuje tyto metody:

    \begin{itemize}
        \item {SAT: Kontruktor třídy, na základě dat, které jsou poslány v parametru, vytváří instance třídy Clause.}
        \item {CalculateWeight: Vrací váhu všech proměnných jako jejich součet.}
        \item {CalculateFulfilled: na základě genomu, který získá jako vstupní parametr vypočítá součet všech splněných klauzulí.}
        \item {CalculateClausesWeight: Vrací váhu použitých proměnných podle genomu, který je předán jako vstupní parametr.}
        \item {ClausesSize: Vrací počet klauzulí.}
    \end{itemize}

    Třída Clause je již reprezentací jednotlivých klauzulí, nese pouze informaci o proměnných, které jsou použitý v dané klauzuli. Implentuje tyto metody:

    \begin{itemize}
        \item {Clause: Konstruktor třídy, inicializuje data.}
        \item {Fulfilled: Vrací bool s informací o splnění klauzule na základě genomu ze vstupních dat.}
        \item {ClauseWeight: Vrací váhu dané klauzule.}
    \end{itemize}

    Jako poslední je třída Specimen, která je reprezentací jedince v populaci genetického algoritmu. Uchovává v sobě informace o genomu daného jedince,
    genom v tomto případě je pole boolů, které reprezentují použití jednotlivých proměnných v řešení SAT problému, také si nese informaci o hodnotě své fitness funkce.
    Instance má take shared\_ptr na instanci SATu. Implementuje tyto metody:

    \begin{itemize}
        \item {Specimen: Konstruktor třídy, inicializuje náhodně genom., díky tomu je zajistěna prvotní diverzita. Také obsahuje copy konstruktor, který vytvoří hlubokou kopii.}
        \item {Crossbreed: Metoda, která slouží ke skřížení 2 jedinců. Na vstupu je jedinec, který předá část svého genomu a také pozice, od které je genom předán.}
        \item {GetGenom: Metoda pro získání genomu, neboť genom je jako privátní parametr.}
        \item {Mutate: Metoda, která s velmi nízkou pravděpodobností zmutuje genom daného jedince.}
        \item {Fitness: Slouží k výpočtu hodnoty fitness funkce, které posleze uloží do veřejného parametru.}
    \end{itemize}

    \subsection{Genetický algoritmus}

    V popisu implementace jsem již popsal některé prvky genetického algoritmu. Nyná si projděme všechny na jednom místě.

    Algoritmus napodobuje evoluci, kterou známe z reálného světa. Genom reprezentuje informaci o každém individuu, každé individuum poté prochází
    evolucí, které mění jeho genom. Algoritmus řídí výběr individuí do další generace podle toho, jak blízko se blíží správnému řešení a to díky
    hodnotě fitness funkce. Při každém evolučném kroku ovšem musíme nějdříve provést selekci nad původní populací, nad tímto výběrem poté provedeme
    operaci křížení a také mutaci.

    \subsubsection{Tournament selection (Turnajový výběr)}

    Jedná se o jednu z metod pro výběr individuí z populace. Podmnožina o určité velikosti $n$ je vybrána náhodně, tuto množinu následně seřadíme podle hodnoty
    fitness funkce. Takto seřazená funkce ještě není náš finální výběr, jednotlivé členy zařadíme do další populace s následující pravděpodobností:
    \begin{itemize}
        \item {1. prvek s pravděpodobností $p$}
        \item {2. prvek s pravděpodobností $p*(1-p)$}
        \item {3. prvek s pravděpodobností $p*(1-p)^2$}
        \item {n. prvek s pravděpodobností $p*(1-p)^{n-1}$}
    \end{itemize}

    Takto opakujeme jednotlivé turnaje dokud nenaplníme novou populaci.

    \subsubsection{Crossbreeding (Křížení)}

    Křížení je forma sdílení genomu mezi individui. Pracuje na pricipu přenesení části genomu z jedince A do genomu jedince B a naopak. Křížení je také podmíněno pravděpodobností.
    V mé implementaci probíhá křížení mezi sousedními jedinci, v rámci určitých optimalizací je ovšem doporučeno zajistit výběr jedinců jiným způsobem, například vybírat jedince, kteří
    mají větší rozdíl v hodnotách fitness funkce. Křížení probíhá pouze jednobodové.

    \subsubsection{Mutation (Mutace)}

    Mutace je změna genomu u jednoho jedince, mutace by se děje s určitou pravděpodobností $p$, tato hodnota by ovšem měla být malá, neboť zpomaluje a zhrošuje výstup algoritmu.

    \subsubsection{Fitness funkce}

    Jedna z hlavních části algoritmu, fitness funkce ohodnocuje jednotlivé jedince, pro každého z nich určuje jeho váhu, váha v našem případě reprezentuje hodnotu toho, jak
    moc se daný genom blíží optimu. Implementace této funkce je důležitá, protože nám zajišťuje správný výstup celého algoritmu.

    \subsection{Adaptační mechanizmy}

    \subsubsection{Elitism (Elitářství)}

    Elitářství zachovává vždy množinu nejlepších jedinců z předchozí generace do další. Takto vybrané jedince nemutujeme ani nekřížíme, díky tomu si zachováme nejlepší prvky do další
    generace. Toto bylo důležité implementovat, neboť algoritmus si neumí jinak udržet nejlepší přechozí výsledek se 100\% pravděpodobností, to poté vede, že se může optimální řešení
    zapomenout mezi generacemi.

    \subsubsection{Mass extinction (Hromadné vymírání)}

    Hromadné vymírání je opět inspirováno přírodou. Hromadné vymírání vezme pár nejsilnějších jedinců a pokud jsou splněné podmínky pro spuštění vymírání, těchto pár jedinců
    postupně mutujeme, aby vytvořili novou populaci. Toto má předejít problémům s uvíznutím v lokálním maximu, abych předešli ztrátě výsledku, minimálně jednoho nejlepšího z předešlé
    populace necháme nezměněného do další generace. Hromadné vymírání mi dále pomohlo v řešení problémů s uváznutím v lokálním maximu.

    \pagebreak

    \section{Měření a zkoumání výsledků}

    \subsection{Testovací prostředí}

    Testování probíhá na zařízení s procesorem od firmy Intel, označení Intel i5 (I5-5257U) s výkonem 2.7 GHz, 2 jádra a 4 vlákna. Operační pamět je 8GB a počítač využívá
    operační systém MacOs Catalina verze 0.15.2 (19C57). Program je napsán v jazyce c++ standard 17, program je kompilován s přepínačem -O3 pro maximální optimalizaci a to za pomocí
    kompilátoru clang.

    \subsection{Data}

    Vzorek dat, která používám k měření, jsou k nalezení na \href{https://moodle-vyuka.cvut.cz/pluginfile.php/168161/mod_assign/intro/wuf-A.zip}{této adrese}. Jednotlivé instance
    použíté v daném grafu budou popsány.

    \subsection{Závislost odchylky od výsledku na velikosti populace a počtu generací}

    Počet generací a i velikost populace ovlivňuje délku běhu, ovšem více generací nebo větší populace nemusí vyřešit všechny problémy, musíme najít kompromis.

    \begin{figure}[H]
        \begin{center}
            \begin{tikzpicture}
                \begin{axis}[
                    ybar,
                    enlargelimits=0.15,
                    legend style={at={(0.5,-0.4)},
                      anchor=north,legend columns=-1},
                    ylabel={Průměrná odchylka},
                    symbolic x coords={50 evolucí, 100 evolucí, 500 evolucí},
                    x tick label style={rotate=45,anchor=east},
                    nodes near coords,
                    nodes near coords align={vertical},
                    ]
                \addplot coordinates {(50 evolucí,11.5733333333333) (100 evolucí,7.626666666666650) (500 evolucí,5.360000000000000)};
                \addplot coordinates {(50 evolucí,8.266666666666650) (100 evolucí,6.23999999999999) (500 evolucí,4.080000000000000)};
                \addplot coordinates {(50 evolucí,6.92) (100 evolucí,4.959999999999990) (500 evolucí,2.480000000000000)};
                \legend{Populace 10, Populace 100, Populace 500}
                \end{axis}
            \end{tikzpicture}
            \caption{Graf odchylky v závyslosti na velikosti populace a počtu generací}
        \end{center}
    \end{figure}

    \begin{figure}[H]
        \begin{center}
            \begin{tikzpicture}
                \begin{semilogyaxis}[
                    ybar,
                    enlargelimits=0.15,
                    legend style={at={(0.5,-0.4)},
                      anchor=north,legend columns=-1},
                    ylabel={Průměrný čas},
                    symbolic x coords={50 evolucí, 100 evolucí, 500 evolucí},
                    x tick label style={rotate=45,anchor=east},
                    nodes near coords,
                    nodes near coords align={vertical},
                    ]
                \addplot coordinates {(50 evolucí,0.046402211270000) (100 evolucí,0.078718636110000) (500 evolucí,0.428356433550000)};
                \addplot coordinates {(50 evolucí,0.380980928010000) (100 evolucí,0.835170514090000) (500 evolucí,3.099806121000000)};
                \addplot coordinates {(50 evolucí,1.843979478740001) (100 evolucí,3.962392227660002) (500 evolucí,18.854921533069991)};
                \legend{Populace 10, Populace 100, Populace 500}
                \end{semilogyaxis}
            \end{tikzpicture}
            \caption{Graf závislosti času na velikosti populace a počtu generací}
        \end{center}
    \end{figure}

    Jako nejlepší poměr mezi rychlostí a přesností je možnost použít 100 evolucí na populaci o velikosti 100 jedinců. Jako druhá nejlepší možnost se ještě jeví
    použítí populace o velikosti 500, ovšem tato možnost je na počet operací náročnější a nepřináší velké zlepšení.

    \subsubsection{Porovnání vývoje ceny podle počtu generací}

    V tomto případě sleduju chování programu na jedné instanci, pokaždé ale s jiným počtem generacím, zde je vývoj cen.

    \begin{figure}[H]
        \begin{center}
            \begin{tikzpicture}
                \begin{semilogxaxis}[
                    ylabel={Hodnota vah},
                    xlabel={Generace},
                    legend style={at={(0.5,-0.1)},
                    anchor=north,legend columns=-1}
                ]
                    \addplot table[x index = {0}, y index = {4}] {data/wuf20-014-50-gen.dat};
                    \addplot table[x index = {0}, y index = {4}] {data/wuf20-014-100-gen.dat};
                    \addplot table[x index = {0}, y index = {4}] {data/wuf20-014-500-gen.dat};
                    \addplot[gray, thin, mark=none] coordinates {(0.1,75) (550,75)};
                    \legend{50 generací, 100 generací, 500 generací, Optimum}
                \end{semilogxaxis}
            \end{tikzpicture}
            \caption{Graf vývoje hodnoty fitness funkce bez využítí elitářství}
        \end{center}
    \end{figure}

    Testovací

    \begin{figure}[H]
        \begin{center}
            \begin{tikzpicture}
                \begin{semilogyaxis}[
                    ylabel={Hodnota fitness funkce},
                    xlabel={Generace},
                    legend style={at={(0.5,-0.1)},
                    anchor=north,legend columns=-1}
                ]
                    \addplot table {data/wuf20-590-A-no-elitism.dat};
                    \addplot table[x index = {0}, y index = {3}] {data/wuf20-590-A-no-elitism.dat};
                    \addplot[gray, thin, mark=none] coordinates {(0,0.513888888888889) (100,0.513888888888889)};
                    \legend{Maximum, Medián, Optimum}
                \end{semilogyaxis}
            \end{tikzpicture}
            \caption{Graf vývoje hodnoty fitness funkce bez využítí elitářství}
        \end{center}
    \end{figure}

    Průběh fitness funkce není v tomto případě vzestupný a algoritmus má problém si udržet nejlepší nalezenou hodnotu. Dokonce i při nalezení optima, může v další generaci tuto informaci
    ztratit, proto implementuju elitářství, které zajistí, že hodnota bude mít tendenci spíše růst. Zde je graf po aplikaci elitářství.

    \begin{figure}[H]
        \begin{center}
            \begin{tikzpicture}
                \begin{semilogyaxis}[
                    ylabel={Hodnota fitness funkce},
                    xlabel={Generace},
                    legend style={at={(0.5,-0.1)},
                    anchor=north,legend columns=-1}
                ]
                    \addplot table {data/wuf20-590-A-elitism.dat};
                    \addplot table[x index = {0}, y index = {3}] {data/wuf20-590-A-elitism.dat};
                    \addplot table[x index = {0}, y index = {2}] {data/wuf20-590-A-elitism.dat};
                    \addplot[gray, thin, mark=none] coordinates {(0,0.513888888888889) (100,0.513888888888889)};
                    \legend{Maximum, Medián, Minimum, Optimum}
                \end{semilogyaxis}
            \end{tikzpicture}
            \caption{Graf vývoje hodnoty fitness funkce s využítím elitářství}
        \end{center}
    \end{figure}

    % Problém batohu (v angličtině Knapsack Problem) je jeden z NP-těžkých problémů. Jeho základem je:
    % \begin{itemize}
    %     \item $n \in \mathbb{N}$ určující počet věcí, které chceme vložit do batohu
    %     \item $M \in \mathbb{N}$ označující kapacitu batohu
    %     \item dále mějme množinu $V = \{v_1, v_2, \dots, v_n\}$ vah věcí
    %     \item a množinu $C = \{c_1, c_2, \dots, c_n\}$ cen věcí
    % \end{itemize}

    % V rámci tohoto úkolu budeme řešit pouze konstruktivní problém, hledáme tedy kombinaci takových předmětů z bathou, které se vejdou do batohu a budou
    % mít zároveň nejvyšší možnou cenu.

    % \section{Implementace}

    % Jako pokročilou metodu jsem zvolil genetický algoritmus. Genetický algoritmus je založen na reálné evoluci, puze velmi zjednodušené. Na začátku
    % algoritmu náhodně vygerenuje populaci, kde každý jedinec má unikátní genom, který je v našem případě definován jako pole boolů, kde se nachazí
    % hodnota pro každý predmět, který může být vložen do batohu. Tímto si sestavíme první generaci, u ní vypočítáme fitness hodnotu pro vsechny genomy v
    % populaci. Fitness funkce nám říká silu určitého genomu, tedy to, jak dobře se blíží výsledku. Díky tomuto může poté začít křížit jednotlivé
    % genomy v populaci, toto se děje s určitou pravděpodobností, dále jestě jednotlivé geny podrobíme náhodné mutaci, tato mutace je ovšem velmi
    % málo pravděpodobná narozdíl od křížení. Tímto způsoben necháme algoritmus projít přes několik generací, kde v každé generaci provedeme křížení,
    % mutaci a také zavedeme prvek elitarismu, kdy z naší populace zachováme 2 nejlepší do další generace.

    % \subsection{Fitness funkce}

    % Fitness funkce má největší dopad na, jak dobré výsledky budeme dostávat z našeho generačního algoritmu. Jako fitness funkci sem zvolil vztah, který je založen
    % na již předpočítaném výsledku. Funkce vychází z tohoto zdroje: http://math.stmarys-ca.edu/wp-content/uploads/2017/07/Christopher-Queen.pdf.

    % \subsection{Výběr individuí}

    % Pro výběr individuí do další generace používám turnajový výběr. Ten funguje tak, že na začátku vybere skupinu z populace, seřadí skupiny od nejlepšího po nejhorší a poté je se snižující
    % se pravděpodobností vybere do další generace. Tento výběr probíhá dokud není naplněna celá nová populace.

    % \section{Analýza významu velikosti populace}

    % Nyní se zaměřím na to jak velikost populace ovlivní běh a přesnost algoritmu. Nastavení pravděpodobnosti křížení a mutace necháme nastaveno na 95\% a 0.01\%.
    % Počet generací je v rámci tohoto testovaní nastaven na 100. Nejdříve se podíváme na řešení pro 30 instancí.

    % \begin{figure}[H]
    %     \begin{center}
    %         \begin{tikzpicture}
    %             \begin{axis}
    %                 \addplot table[ignore chars={(,)},col sep=comma] {data/pop500.dat};
    %                 \addplot table[ignore chars={(,)},col sep=comma, x index = {0}, y index = {2}] {data/pop500.dat};
    %                 \addplot table[ignore chars={(,)},col sep=comma, x index = {0}, y index = {3}] {data/pop500.dat};
    %             \end{axis}
    %         \end{tikzpicture}
    %         \caption{Velikost populace je 500 individuí}
    %     \end{center}
    % \end{figure}

    % \begin{figure}[H]
    %     \begin{center}
    %         \begin{tikzpicture}
    %             \begin{axis}
    %                 \addplot table[ignore chars={(,)},col sep=comma] {data/pop200.dat};
    %                 \addplot table[ignore chars={(,)},col sep=comma, x index = {0}, y index = {2}] {data/pop200.dat};
    %                 \addplot table[ignore chars={(,)},col sep=comma, x index = {0}, y index = {3}] {data/pop200.dat};
    %             \end{axis}
    %         \end{tikzpicture}
    %         \caption{Velikost populace je 200 individuí}
    %     \end{center}
    % \end{figure}

    % \begin{figure}[H]
    %     \begin{center}
    %         \begin{tikzpicture}
    %             \begin{axis}
    %                 \addplot table[ignore chars={(,)},col sep=comma] {data/pop1000.dat};
    %                 \addplot table[ignore chars={(,)},col sep=comma, x index = {0}, y index = {2}] {data/pop1000.dat};
    %                 \addplot table[ignore chars={(,)},col sep=comma, x index = {0}, y index = {3}] {data/pop1000.dat};
    %             \end{axis}
    %         \end{tikzpicture}
    %         \caption{Velikost populace je 1000 individuí}
    %     \end{center}
    % \end{figure}

    % \begin{figure}[H]
    %     \begin{center}
    %         \begin{tikzpicture}
    %             \begin{axis}
    %                 \addplot table[ignore chars={(,)},col sep=comma] {data/pop50.dat};
    %                 \addplot table[ignore chars={(,)},col sep=comma, x index = {0}, y index = {2}] {data/pop50.dat};
    %                 \addplot table[ignore chars={(,)},col sep=comma, x index = {0}, y index = {3}] {data/pop50.dat};
    %             \end{axis}
    %         \end{tikzpicture}
    %         \caption{Velikost populace je 50 individuí}
    %     \end{center}
    % \end{figure}

    % Z grafů je patrné, že čím větší populaci máme, tím větší diverzita v populaci je. Jak je ovšem vidět, všechny populace se dostaly
    % ke správnému výsledku.

    % Teď se podíváme jak se věci změní pokud zvýšíme počet instancí na 40.

    % \begin{figure}[H]
    %     \begin{center}
    %         \begin{tikzpicture}
    %             \begin{axis}
    %                 \addplot table[ignore chars={(,)},col sep=comma] {data/pop50_40.dat};
    %                 \addplot table[ignore chars={(,)},col sep=comma, x index = {0}, y index = {2}] {data/pop50_40.dat};
    %                 \addplot table[ignore chars={(,)},col sep=comma, x index = {0}, y index = {3}] {data/pop50_40.dat};
    %             \end{axis}
    %         \end{tikzpicture}
    %         \caption{Velikost populace je 50 individuí}
    %     \end{center}
    % \end{figure}

    % \begin{figure}[H]
    %     \begin{center}
    %         \begin{tikzpicture}
    %             \begin{axis}
    %                 \addplot table[ignore chars={(,)},col sep=comma] {data/pop200_40.dat};
    %                 \addplot table[ignore chars={(,)},col sep=comma, x index = {0}, y index = {2}] {data/pop200_40.dat};
    %                 \addplot table[ignore chars={(,)},col sep=comma, x index = {0}, y index = {3}] {data/pop200_40.dat};
    %             \end{axis}
    %         \end{tikzpicture}
    %         \caption{Velikost populace je 200 individuí}
    %     \end{center}
    % \end{figure}

    % \begin{figure}[H]
    %     \begin{center}
    %         \begin{tikzpicture}
    %             \begin{axis}
    %                 \addplot table[ignore chars={(,)},col sep=comma] {data/pop500_40.dat};
    %                 \addplot table[ignore chars={(,)},col sep=comma, x index = {0}, y index = {2}] {data/pop500_40.dat};
    %                 \addplot table[ignore chars={(,)},col sep=comma, x index = {0}, y index = {3}] {data/pop500_40.dat};
    %             \end{axis}
    %         \end{tikzpicture}
    %         \caption{Velikost populace je 500 individuí}
    %     \end{center}
    % \end{figure}

    % \begin{figure}[H]
    %     \begin{center}
    %         \begin{tikzpicture}
    %             \begin{axis}
    %                 \addplot table[ignore chars={(,)},col sep=comma] {data/pop1000_40.dat};
    %                 \addplot table[ignore chars={(,)},col sep=comma, x index = {0}, y index = {2}] {data/pop1000_40.dat};
    %                 \addplot table[ignore chars={(,)},col sep=comma, x index = {0}, y index = {3}] {data/pop1000_40.dat};
    %             \end{axis}
    %         \end{tikzpicture}
    %         \caption{Velikost populace je 1000 individuí}
    %     \end{center}
    % \end{figure}

    % Jak je vidět pro vyšší počet instancí už trvá déle se dostat ke správnému výsledku a našich 100 generací pro to tedy nestačí.

    % \section{Význam počtu evolucí}

    % Z posledních grafů je vidět, že naše generace se postupně přibližovaly výsledku. Zkusme se nyní zaměřit na to jestli počet generací bude ovlivňovat to,
    % jak moc se výsledku přiblížíme. Populaci zkusíme nechat na 500 a poté ještě zkusíme rozdíl mezi 500 a 1000 individuí.

    % \begin{figure}[H]
    %     \begin{center}
    %         \begin{tikzpicture}
    %             \begin{axis}
    %                 \addplot table[ignore chars={(,)},col sep=comma] {data/evo100_pop500_40.dat};
    %                 \addplot table[ignore chars={(,)},col sep=comma, x index = {0}, y index = {2}] {data/evo100_pop500_40.dat};
    %                 \addplot table[ignore chars={(,)},col sep=comma, x index = {0}, y index = {3}] {data/evo100_pop500_40.dat};
    %             \end{axis}
    %         \end{tikzpicture}
    %         \caption{Velikost populace je 500 individuí, počet evolucí je 100}
    %     \end{center}
    % \end{figure}

    % \begin{figure}[H]
    %     \begin{center}
    %         \begin{tikzpicture}
    %             \begin{axis}
    %                 \addplot table[ignore chars={(,)},col sep=comma] {data/evo250_pop500_40.dat};
    %                 \addplot table[ignore chars={(,)},col sep=comma, x index = {0}, y index = {2}] {data/evo250_pop500_40.dat};
    %                 \addplot table[ignore chars={(,)},col sep=comma, x index = {0}, y index = {3}] {data/evo250_pop500_40.dat};
    %             \end{axis}
    %         \end{tikzpicture}
    %         \caption{Velikost populace je 500 individuí, počet evolucí je 250}
    %     \end{center}
    % \end{figure}

    % \begin{figure}[H]
    %     \begin{center}
    %         \begin{tikzpicture}
    %             \begin{axis}
    %                 \addplot table[ignore chars={(,)},col sep=comma] {data/evo500_pop500_40.dat};
    %                 \addplot table[ignore chars={(,)},col sep=comma, x index = {0}, y index = {2}] {data/evo500_pop500_40.dat};
    %                 \addplot table[ignore chars={(,)},col sep=comma, x index = {0}, y index = {3}] {data/evo500_pop500_40.dat};
    %             \end{axis}
    %         \end{tikzpicture}
    %         \caption{Velikost populace je 500 individuí, počet evolucí je 500}
    %     \end{center}
    % \end{figure}

    % \begin{figure}[H]
    %     \begin{center}
    %         \begin{tikzpicture}
    %             \begin{axis}
    %                 \addplot table[ignore chars={(,)},col sep=comma] {data/evo1000_pop500_40.dat};
    %                 \addplot table[ignore chars={(,)},col sep=comma, x index = {0}, y index = {2}] {data/evo1000_pop500_40.dat};
    %                 \addplot table[ignore chars={(,)},col sep=comma, x index = {0}, y index = {3}] {data/evo1000_pop500_40.dat};
    %             \end{axis}
    %         \end{tikzpicture}
    %         \caption{Velikost populace je 500 individuí, počet evolucí je 1000}
    %     \end{center}
    % \end{figure}

    % Pro populaci o velikosti 500 to vypadá, že se dostáváme do nějakého lokálního maxima, zkusíme zvýšit počet individuí, pokud stále uvázneme v určitém
    % maximu musí zvýšit tlak na evoluci.

    % \begin{figure}[H]
    %     \begin{center}
    %         \begin{tikzpicture}
    %             \begin{axis}
    %                 \addplot table[ignore chars={(,)},col sep=comma] {data/evo100_pop1000_40.dat};
    %                 \addplot table[ignore chars={(,)},col sep=comma, x index = {0}, y index = {2}] {data/evo100_pop1000_40.dat};
    %                 \addplot table[ignore chars={(,)},col sep=comma, x index = {0}, y index = {3}] {data/evo100_pop1000_40.dat};
    %             \end{axis}
    %         \end{tikzpicture}
    %         \caption{Velikost populace je 1000 individuí, počet evolucí je 100}
    %     \end{center}
    % \end{figure}

    % \begin{figure}[H]
    %     \begin{center}
    %         \begin{tikzpicture}
    %             \begin{axis}
    %                 \addplot table[ignore chars={(,)},col sep=comma] {data/evo250_pop1000_40.dat};
    %                 \addplot table[ignore chars={(,)},col sep=comma, x index = {0}, y index = {2}] {data/evo250_pop1000_40.dat};
    %                 \addplot table[ignore chars={(,)},col sep=comma, x index = {0}, y index = {3}] {data/evo250_pop1000_40.dat};
    %             \end{axis}
    %         \end{tikzpicture}
    %         \caption{Velikost populace je 1000 individuí, počet evolucí je 250}
    %     \end{center}
    % \end{figure}

    % \begin{figure}[H]
    %     \begin{center}
    %         \begin{tikzpicture}
    %             \begin{axis}
    %                 \addplot table[ignore chars={(,)},col sep=comma] {data/evo500_pop1000_40.dat};
    %                 \addplot table[ignore chars={(,)},col sep=comma, x index = {0}, y index = {2}] {data/evo500_pop1000_40.dat};
    %                 \addplot table[ignore chars={(,)},col sep=comma, x index = {0}, y index = {3}] {data/evo500_pop1000_40.dat};
    %             \end{axis}
    %         \end{tikzpicture}
    %         \caption{Velikost populace je 1000 individuí, počet evolucí je 500}
    %     \end{center}
    % \end{figure}

    % \begin{figure}[H]
    %     \begin{center}
    %         \begin{tikzpicture}
    %             \begin{axis}
    %                 \addplot table[ignore chars={(,)},col sep=comma] {data/evo1000_pop1000_40.dat};
    %                 \addplot table[ignore chars={(,)},col sep=comma, x index = {0}, y index = {2}] {data/evo1000_pop1000_40.dat};
    %                 \addplot table[ignore chars={(,)},col sep=comma, x index = {0}, y index = {3}] {data/evo1000_pop1000_40.dat};
    %             \end{axis}
    %         \end{tikzpicture}
    %         \caption{Velikost populace je 1000 individuí, počet evolucí je 1000}
    %     \end{center}
    % \end{figure}

    % Je vidět, že algoritmus má problém se změnit a uvázne v lokálním maximu, ze kterého je težké se dostat, proto se zaměřme na selekční tlak.

    % \section{Turnajová metoda výběru}

    % Podívejme se na to jak výběr našich individuí ovlivní evoluci naší populace. Nyní jsme do turnaje brali polovinu populace. Zkusme tedy měnit velikost
    % turnaje, jak se toto projeví na výsledcích. Nastavení populace necháme na 1000 a počet generací také, na takto velkých datech nejjednodušeji uvidíme, kdy uvázmene
    % v lokálním extrému.

    % \begin{figure}[H]
    %     \begin{center}
    %         \begin{tikzpicture}
    %             \begin{axis}
    %                 \addplot table[ignore chars={(,)},col sep=comma] {data/tour20.dat};
    %                 \addplot table[ignore chars={(,)},col sep=comma, x index = {0}, y index = {2}] {data/tour20.dat};
    %                 \addplot table[ignore chars={(,)},col sep=comma, x index = {0}, y index = {3}] {data/tour20.dat};
    %             \end{axis}
    %         \end{tikzpicture}
    %         \caption{Velikost turnaje je 20}
    %     \end{center}
    % \end{figure}

    % \begin{figure}[H]
    %     \begin{center}
    %         \begin{tikzpicture}
    %             \begin{axis}
    %                 \addplot table[ignore chars={(,)},col sep=comma] {data/tour100.dat};
    %                 \addplot table[ignore chars={(,)},col sep=comma, x index = {0}, y index = {2}] {data/tour100.dat};
    %                 \addplot table[ignore chars={(,)},col sep=comma, x index = {0}, y index = {3}] {data/tour100.dat};
    %             \end{axis}
    %         \end{tikzpicture}
    %         \caption{Velikost turnaje je 100}
    %     \end{center}
    % \end{figure}

    % \begin{figure}[H]
    %     \begin{center}
    %         \begin{tikzpicture}
    %             \begin{axis}
    %                 \addplot table[ignore chars={(,)},col sep=comma] {data/tour250.dat};
    %                 \addplot table[ignore chars={(,)},col sep=comma, x index = {0}, y index = {2}] {data/tour250.dat};
    %                 \addplot table[ignore chars={(,)},col sep=comma, x index = {0}, y index = {3}] {data/tour250.dat};
    %             \end{axis}
    %         \end{tikzpicture}
    %         \caption{Velikost turnaje je 250}
    %     \end{center}
    % \end{figure}

    % \begin{figure}[H]
    %     \begin{center}
    %         \begin{tikzpicture}
    %             \begin{axis}
    %                 \addplot table[ignore chars={(,)},col sep=comma] {data/tour500.dat};
    %                 \addplot table[ignore chars={(,)},col sep=comma, x index = {0}, y index = {2}] {data/tour500.dat};
    %                 \addplot table[ignore chars={(,)},col sep=comma, x index = {0}, y index = {3}] {data/tour500.dat};
    %             \end{axis}
    %         \end{tikzpicture}
    %         \caption{Velikost turnaje je 500}
    %     \end{center}
    % \end{figure}

    % \begin{figure}[H]
    %     \begin{center}
    %         \begin{tikzpicture}
    %             \begin{axis}
    %                 \addplot table[ignore chars={(,)},col sep=comma] {data/tour750.dat};
    %                 \addplot table[ignore chars={(,)},col sep=comma, x index = {0}, y index = {2}] {data/tour750.dat};
    %                 \addplot table[ignore chars={(,)},col sep=comma, x index = {0}, y index = {3}] {data/tour750.dat};
    %             \end{axis}
    %         \end{tikzpicture}
    %         \caption{Velikost turnaje je 750}
    %     \end{center}
    % \end{figure}

    % Jak je vidět, nejoptimálnější byla velikost turnaje 100.

    % \section{Závěr}

    % Nejdříve jak je náš algoritmus přesný, v grafech jsme viděli, jak se jednotlivé generace blíží k hodnocení fitness funkce rovno jedné. Pokud se generace rovná jedné,
    % jedná se o správný výsledek. Také je důležité sledovat medián, který nám udává, jak je na tom vyšší polovina naší populace a jeho hodnota čím větší, tím lepší. Co se týče
    % měření času, zde je statistika pro 500 ruzných řešení batohu pro 40 instancí, populace i počet evolucí je shodný a roven 1000, velikost turnaje 20, tyto data ukazují nejdelší běh.
    % Čas byl měřen na 2,7 GHz Dual-Core Intel Core i5 (I5-5257U), průměrný čas pro 500 problémů je 2.2456 sekund a průměrná odchylka od řešení je 0.005332996117239.
    % Odchylka bude se zmenšovat se snižujícím se počtem instancí díky větší pravděpodobnosti trefy do správného řešení.
\end{document}