% arara: pdflatex
% arara: pdflatex

\documentclass{article}
\usepackage[utf8]{inputenc}
\usepackage[czech]{babel}
\usepackage{amssymb}
\usepackage{pgfplots}
\usepackage{float}
\usepackage{hyperref}
\usepackage{todonotes}


\title{Řešení problému vážené splnitelnosti booleovské formule pokročilou iterativní metodou}
\author{Tomáš Starý (staryto5)}

\begin{document}
    \maketitle
    \tableofcontents
    \pagebreak

    \section{Zadání úlohy}

    Podrobné zadání úlohy je na \href{https://moodle-vyuka.cvut.cz/mod/assign/view.php?id=48355}{této adrese}. Jedná se o řešení 3SAT problému pomocí některé z pokročlých heuristik:

    \begin{itemize}
        \item {simulované ochlazování}
        \item {genetický algoritmus}
        \item {tabu prohledávání}
    \end{itemize}

    Po nasazení heuristiky je potřeba ověřit její vhodnost a to pomocí experimentálního vyhodnocení.

    \subsection{Popis řešeného problému}

    Dána booleovská formule $F$ o $n$ proměnných, $X = (x_1, x_2,\dots,x_n)$ v konjunktivní normální formě (CNF).
    Dále jsou dány celočíselné kladné váhy těchto $n$ proměnných $W = (w_1, w_2,\dots, w_n)$. Nalezněte přiřazení $Y=(y_1, y_2,\dots, y_n)$
    proměnných $x_1, x_2,\dots, x_n$, takové, že $F(Y) = 1$ a součet $S$ vah proměnných nastavených do 1 je maximální.

    Omezte se na vážený 3-SAT problém, kde každá klauzule má právě 3 proměnné. Složitost problému je stejná, implementace a klasifikace instancí jsou jednodušší.

    \section{Zpracování a implementace řešení}

    Jako nejdříve sem začal parserem na zpracování souboru, u kterého sem kontroloval hodnoty, které ze souboru získal. Následovalo načtení dat do konečné struktury.

    Posléze jsem implementoval základní genetický algoritmus s křížením, mutací a výběrem jedinců do další generace pomocí turnajového výběru. Na menších instancích sem
    zkoušel chování programu a upravoval jednotlivé části, abych dosáhl optimálního výsledku.

    Následovalo přidání adaptačních mechanizmů, které měli nadále zlepšit fungování algoritmu a jeho přesnější výsledky. Tyto mechanizmy sem poté podrobil měření jejich přínosu
    pro algoritmus.

    \subsection{Struktura programu}

    Program je psaný v jazyce C++, kompilován s maximální optimalizací (přepínač -O3) pomocí clang kompilátoru. Pro reprezentaci klauzulí a celého SAT problému byly
    použity třídy (Clause, SAT). Parser je poté samostatná funkce, které sestavuje třídu SATSolver, která vytváří ostatní objekty. Pro účely genetického algoritmu
    jsem implementoval ještě třídu Specimen, která reprezentuje jednotlivé členy v populaci.

    SATSolver je třída, která má na starosti vše okolo inicializace a řešení SAT problému. Nese v sobě informace o populaci, instanci SAT problému, velikosti populace,
    počtu klauzulí a počtu proměnných. Implementuje tyto metody:

    \begin{itemize}
        \item {SATSolver: Konstruktor třídy, jako vstupní data má pole vah jednotlivých proměnných a pole jednotlivých klauzulí. Na základě těchto dat vytvoří potřebnou
                datovou strukturu. Také vytáří základní populaci.}
        \item {Solve: Tato metoda slouží k spuštění a provedení samotné evoluce, která je vlastním řešením problému.}
        \item {Crossbreeding: Privátní metoda, má na starost samotné křížení mezi jednotlivými členy populace.}
        \item {Mutation: Také privítní metoda, řeši mutaci jednotlivých členů populace.}
        \item {MassExtenction: Metoda, která v případě splnění podmínky spustí hromadné vymírání a sestavení nové populace.}
    \end{itemize}

    Třída SAT obsahuje všechny důležité informace o SAT problému. Nese informace o váhách proměnných a klauzulích, které dostane při zakládaní od SATSolver. Jednotlivé instance
    třídy Clause si ovšem třída zakládá sama z dat, které jsou jí předány v konstruktoru. Obsahuje tyto metody:

    \begin{itemize}
        \item {SAT: Kontruktor třídy, na základě dat, které jsou poslány v parametru, vytváří instance třídy Clause.}
        \item {CalculateWeight: Vrací váhu všech proměnných jako jejich součet.}
        \item {CalculateFulfilled: na základě genomu, který získá jako vstupní parametr vypočítá součet všech splněných klauzulí.}
        \item {CalculateClausesWeight: Vrací váhu použitých proměnných podle genomu, který je předán jako vstupní parametr.}
        \item {ClausesSize: Vrací počet klauzulí.}
    \end{itemize}

    Třída Clause je již reprezentací jednotlivých klauzulí, nese pouze informaci o proměnných, které jsou použitý v dané klauzuli. Implentuje tyto metody:

    \begin{itemize}
        \item {Clause: Konstruktor třídy, inicializuje data.}
        \item {Fulfilled: Vrací bool s informací o splnění klauzule na základě genomu ze vstupních dat.}
        \item {ClauseWeight: Vrací váhu dané klauzule.}
    \end{itemize}

    Jako poslední je třída Specimen, která je reprezentací jedince v populaci genetického algoritmu. Uchovává v sobě informace o genomu daného jedince,
    genom v tomto případě je pole boolů, které reprezentují použití jednotlivých proměnných v řešení SAT problému, také si nese informaci o hodnotě své fitness funkce.
    Instance má take shared\_ptr na instanci SATu. Implementuje tyto metody:

    \begin{itemize}
        \item {Specimen: Konstruktor třídy, inicializuje náhodně genom., díky tomu je zajistěna prvotní diverzita. Také obsahuje copy konstruktor, který vytvoří hlubokou kopii.}
        \item {Crossbreed: Metoda, která slouží ke skřížení 2 jedinců. Na vstupu je jedinec, který předá část svého genomu a také pozice, od které je genom předán.}
        \item {GetGenom: Metoda pro získání genomu, neboť genom je jako privátní parametr.}
        \item {Mutate: Metoda, která s pravděpodobností ze vstupu zmutuje genom daného jedince.}
        \item {Fitness: Slouží k výpočtu hodnoty fitness funkce, které posleze uloží do veřejného parametru.}
    \end{itemize}

    \subsection{Genetický algoritmus}

    V popisu implementace jsem již popsal některé prvky genetického algoritmu. Nyná si projděme všechny na jednom místě.

    Algoritmus napodobuje evoluci, kterou známe z reálného světa. Genom reprezentuje informaci o každém individuu, každé individuum poté prochází
    evolucí, které mění jeho genom. Algoritmus řídí výběr individuí do další generace podle toho, jak blízko se blíží správnému řešení a to díky
    hodnotě fitness funkce. Při každém evolučném kroku ovšem musíme nějdříve provést selekci nad původní populací, nad tímto výběrem poté provedeme
    operaci křížení a také mutaci.

    \subsubsection{Tournament selection (Turnajový výběr)}

    Jedná se o jednu z metod pro výběr individuí z populace. Podmnožina o určité velikosti $n$ je vybrána náhodně, tuto množinu následně seřadíme podle hodnoty
    fitness funkce. Takto seřazená funkce ještě není náš finální výběr, jednotlivé členy zařadíme do další populace s následující pravděpodobností:
    \begin{itemize}
        \item {1. prvek s pravděpodobností $p$}
        \item {2. prvek s pravděpodobností $p*(1-p)$}
        \item {3. prvek s pravděpodobností $p*(1-p)^2$}
        \item {n. prvek s pravděpodobností $p*(1-p)^{n-1}$}
    \end{itemize}

    Takto opakujeme jednotlivé turnaje dokud nenaplníme novou populaci.

    \subsubsection{Crossbreeding (Křížení)}

    Křížení je forma sdílení genomu mezi individui. Pracuje na pricipu přenesení části genomu z jedince A do genomu jedince B a naopak. Křížení je také podmíněno pravděpodobností.
    V mé implementaci probíhá křížení mezi sousedními jedinci, v rámci určitých optimalizací je ovšem doporučeno zajistit výběr jedinců jiným způsobem, například vybírat jedince, kteří
    mají větší rozdíl v hodnotách fitness funkce. Křížení probíhá pouze jednobodové.

    \subsubsection{Mutation (Mutace)}

    Mutace je změna genomu u jednoho jedince, mutace by se děje s určitou pravděpodobností $p$, tato hodnota by ovšem měla být malá, neboť zpomaluje a zhrošuje výstup algoritmu.

    \subsubsection{Fitness funkce}

    Jedna z hlavních části algoritmu, fitness funkce ohodnocuje jednotlivé jedince, pro každého z nich určuje jeho váhu, váha v našem případě reprezentuje hodnotu toho, jak
    moc se daný genom blíží optimu. Implementace této funkce je důležitá, protože nám zajišťuje správný výstup celého algoritmu.

    \subsection{Adaptační mechanizmy}

    \subsubsection{Elitism (Elitářství)}

    Elitářství zachovává vždy množinu nejlepších jedinců z předchozí generace do další. Takto vybrané jedince nemutujeme ani nekřížíme, díky tomu si zachováme nejlepší prvky do další
    generace. Toto bylo důležité implementovat, neboť algoritmus si neumí jinak udržet nejlepší přechozí výsledek se 100\% pravděpodobností, to poté vede, že se může optimální řešení
    zapomenout mezi generacemi.

    \subsubsection{Mass extinction (Hromadné vymírání)}

    Hromadné vymírání je opět inspirováno přírodou. Hromadné vymírání vezme pár nejsilnějších jedinců a pokud jsou splněné podmínky pro spuštění vymírání, těchto pár jedinců
    postupně mutujeme, aby vytvořili novou populaci. Toto má předejít problémům s uvíznutím v lokálním maximu, abych předešli ztrátě výsledku, minimálně jednoho nejlepšího z předešlé
    populace necháme nezměněného do další generace. Hromadné vymírání mi dále pomohlo v řešení problémů s uváznutím v lokálním maximu.

    Ve svém řešení sem implementoval hromadné vymírání tak, že jsem jej svázal s mediánem fitness funkce napříč populací, pokud medián dosáhne stejné hodnoty jako je maximum, je hromadné
    vymírání spuštěno. Při tomto vymírání mají nový jedinci 95\% šanci na mutaci.

    \pagebreak

    \section{Měření a zkoumání výsledků}

    \subsection{Testovací prostředí}

    Testování probíhá na zařízení s procesorem od firmy Intel, označení Intel i5 (I5-5257U) s výkonem 2.7 GHz, 2 jádra a 4 vlákna. Operační pamět je 8GB a počítač využívá
    operační systém MacOs Catalina verze 0.15.2 (19C57). Program je napsán v jazyce c++ standard 17, program je kompilován s přepínačem -O3 pro maximální optimalizaci a to za pomocí
    kompilátoru clang.

    \subsection{Data}

    Vzorek dat, která používám k měření, jsou k nalezení na \href{https://moodle-vyuka.cvut.cz/pluginfile.php/168161/mod_assign/intro/wuf-A.zip}{této adrese}. Jednotlivé instance
    použíté v daném grafu budou popsány.

    \subsection{Závislost odchylky od výsledku na velikosti populace a počtu generací}

    Počet generací a i velikost populace ovlivňuje délku běhu, ovšem více generací nebo větší populace nemusí vyřešit všechny problémy, musíme najít kompromis.

    \begin{figure}[H]
        \begin{center}
            \begin{tikzpicture}
                \begin{axis}[
                    ybar,
                    enlargelimits=0.15,
                    legend style={at={(0.5,-0.4)},
                      anchor=north,legend columns=-1},
                    ylabel={Průměrná odchylka},
                    symbolic x coords={50 evolucí, 100 evolucí, 500 evolucí},
                    x tick label style={rotate=45,anchor=east},
                    nodes near coords,
                    nodes near coords align={vertical},
                    ]
                \addplot coordinates {(50 evolucí,11.5733333333333) (100 evolucí,7.626666666666650) (500 evolucí,5.360000000000000)};
                \addplot coordinates {(50 evolucí,8.266666666666650) (100 evolucí,6.23999999999999) (500 evolucí,4.080000000000000)};
                \addplot coordinates {(50 evolucí,6.92) (100 evolucí,4.959999999999990) (500 evolucí,2.480000000000000)};
                \legend{Populace 10, Populace 100, Populace 500}
                \end{axis}
            \end{tikzpicture}
            \caption{Graf odchylky v závyslosti na velikosti populace a počtu generací}
        \end{center}
    \end{figure}

    \begin{figure}[H]
        \begin{center}
            \begin{tikzpicture}
                \begin{semilogyaxis}[
                    ybar,
                    enlargelimits=0.15,
                    legend style={at={(0.5,-0.4)},
                      anchor=north,legend columns=-1},
                    ylabel={Průměrný čas},
                    symbolic x coords={50 evolucí, 100 evolucí, 500 evolucí},
                    x tick label style={rotate=45,anchor=east},
                    nodes near coords,
                    nodes near coords align={vertical},
                    ]
                \addplot coordinates {(50 evolucí,0.046402211270000) (100 evolucí,0.078718636110000) (500 evolucí,0.428356433550000)};
                \addplot coordinates {(50 evolucí,0.380980928010000) (100 evolucí,0.835170514090000) (500 evolucí,3.099806121000000)};
                \addplot coordinates {(50 evolucí,1.843979478740001) (100 evolucí,3.962392227660002) (500 evolucí,18.854921533069991)};
                \legend{Populace 10, Populace 100, Populace 500}
                \end{semilogyaxis}
            \end{tikzpicture}
            \caption{Graf závislosti času na velikosti populace a počtu generací}
        \end{center}
    \end{figure}

    Jako nejlepší poměr mezi rychlostí a přesností je možnost použít 100 evolucí na populaci o velikosti 100 jedinců. Jako druhá nejlepší možnost se ještě jeví
    použítí populace o velikosti 500, ovšem tato možnost je na počet operací náročnější a nepřináší velké zlepšení.

    \pagebreak

    \subsubsection{Porovnání vývoje ceny podle počtu generací}

    V tomto případě sleduju chování programu na jedné instanci, pokaždé ale s jiným počtem generacím, zde je vývoj cen.

    \begin{figure}[H]
        \begin{center}
            \begin{tikzpicture}
                \begin{semilogxaxis}[
                    ylabel={Hodnota vah},
                    xlabel={Generace},
                    legend style={at={(0.5,-0.1)},
                    anchor=north,legend columns=-1}
                ]
                    \addplot table[x index = {0}, y index = {4}] {data/wuf20-014-50-gen.dat};
                    \addplot table[x index = {0}, y index = {4}] {data/wuf20-014-100-gen.dat};
                    \addplot table[x index = {0}, y index = {4}] {data/wuf20-014-500-gen.dat};
                    \addplot[gray, thin, mark=none] coordinates {(0.9,75) (550,75)};
                    \legend{50 generací, 100 generací, 500 generací, Optimum}
                \end{semilogxaxis}
            \end{tikzpicture}
            \caption{Graf vývoje hodnoty vah}
        \end{center}
    \end{figure}

    Testovací soubor byl ze sady wuf20-88-A, konkrétně wuf20-014-A. Optímální váha má hodnotu 75 a
    jak je z grafu vidět všechny tři běhy se už v raných fázích výsledku přiblíží, běhy na 100 a 500 generací dokonce výsledku dosáhnou a tuto informaci si drží po celou dobu běhu
    a to díky elitarismu.

    \subsubsection{Přínos elitarismu}

    V druhé části tohoto dokumentu jsem psal, že implementovat elitarismu je velmi důležité. Toto tvrzení bych měl podložit měřeními, proto se nyní podívejme na graf běhu
    programu s a bez elitarismu. Měření probíhá na souboru wuf20-590-A ze sady wuf20-80-A.

    \begin{figure}[H]
        \begin{center}
            \begin{tikzpicture}
                \begin{semilogyaxis}[
                    ylabel={Hodnota fitness funkce},
                    xlabel={Generace},
                    legend style={at={(0.5,-0.1)},
                    anchor=north,legend columns=-1}
                ]
                    \addplot table {data/wuf20-590-A-no-elitism.dat};
                    \addplot table[x index = {0}, y index = {3}] {data/wuf20-590-A-no-elitism.dat};
                    \addplot[gray, thin, mark=none] coordinates {(0,0.513888888888889) (100,0.513888888888889)};
                    \legend{Maximum, Medián, Optimum}
                \end{semilogyaxis}
            \end{tikzpicture}
            \caption{Graf vývoje hodnoty fitness funkce bez využítí elitářství}
        \end{center}
    \end{figure}

    Průběh fitness funkce není v tomto případě vzestupný a algoritmus má problém si udržet nejlepší nalezenou hodnotu. Dokonce i při nalezení optima, může v další generaci tuto informaci
    ztratit, proto implementuju elitářství, které zajistí, že hodnota bude mít tendenci spíše růst. Zde je graf po aplikaci elitářství.

    \begin{figure}[H]
        \begin{center}
            \begin{tikzpicture}
                \begin{semilogyaxis}[
                    ylabel={Hodnota fitness funkce},
                    xlabel={Generace},
                    legend style={at={(0.5,-0.1)},
                    anchor=north,legend columns=-1}
                ]
                    \addplot table {data/wuf20-590-A-elitism.dat};
                    \addplot table[x index = {0}, y index = {3}] {data/wuf20-590-A-elitism.dat};
                    \addplot table[x index = {0}, y index = {2}] {data/wuf20-590-A-elitism.dat};
                    \addplot[gray, thin, mark=none] coordinates {(0,0.513888888888889) (100,0.513888888888889)};
                    \legend{Maximum, Medián, Minimum, Optimum}
                \end{semilogyaxis}
            \end{tikzpicture}
            \caption{Graf vývoje hodnoty fitness funkce s využítím elitářství}
        \end{center}
    \end{figure}

    Po aplikaci elitářství si algoritmus umí udržet nejlepší hodnotu, nepřijde tedy například o optimum, pokud jej najde. Také je vidět, že i medián a v některých případech i minimum
    se blíží maximální hodnotě. Toto přináší jiný problém, možnost uváznutí v lokálním maximu, proto jsem také implementoval hromadné vymírání, které tento problém částečně odstraňuje.

    \subsubsection{Hromadné vymírání, jak se promítne na hodnoty fitness funkce?}

    To to je další otázka, která by měla být určitě zodpovězena, připravil jsem si proto opět 2 grafy, které ukazují vývoj hodnot fitness funkce při použítí či nepoužítí fitness funkce.
    Testování proběhlo na souboru wuf20-0981-A z datasetu wuf20-88-A.

    \begin{figure}[H]
        \begin{center}
            \begin{tikzpicture}
                \begin{semilogyaxis}[
                    ylabel={Hodnota fitness funkce},
                    xlabel={Generace},
                    legend style={at={(0.5,-0.1)},
                    anchor=north,legend columns=-1}
                ]
                    \addplot table {data/wuf20-0981-A-without-MA.dat};
                    \addplot table[x index = {0}, y index = {3}] {data/wuf20-0981-A-without-MA.dat};
                    \addplot table[x index = {0}, y index = {2}] {data/wuf20-0981-A-without-MA.dat};
                    \addplot[gray, thin, mark=none] coordinates {(0,0.536231884057971) (100,0.536231884057971)};
                    \legend{Maximum, Medián, Minimum, Optimum}
                \end{semilogyaxis}
            \end{tikzpicture}
            \caption{Graf vývoje hodnoty fitness funkce bez hromadného vymírání}
        \end{center}
    \end{figure}

    \begin{figure}[H]
        \begin{center}
            \begin{tikzpicture}
                \begin{semilogyaxis}[
                    ylabel={Hodnota fitness funkce},
                    xlabel={Generace},
                    legend style={at={(0.5,-0.1)},
                    anchor=north,legend columns=-1}
                ]
                    \addplot table {data/wuf20-0981-A-with-MA.dat};
                    \addplot table[x index = {0}, y index = {3}] {data/wuf20-0981-A-with-MA.dat};
                    \addplot table[x index = {0}, y index = {2}] {data/wuf20-0981-A-with-MA.dat};
                    \addplot[gray, thin, mark=none] coordinates {(0,0.536231884057971) (100,0.536231884057971)};
                    \legend{Maximum, Medián, Minimum, Optimum}
                \end{semilogyaxis}
            \end{tikzpicture}
            \caption{Graf vývoje hodnoty fitness funkce s hromadným vymíráním}
        \end{center}
    \end{figure}

    Z grafů je hned patrný rozdíl, algoritmus bez hromadného vymírání přebírá postupně hodnoty nejlepšího jedince a už nemá tendenci se měnit
    narozdíl od algoritmu s hromadným výmíráním, který se při takové konvergenci snaží zamezit uváznutí v lokálním maximu pomocí generování častečně nové populace.

    \pagebreak

    \section{Závěr}

    Cílem této práce bylo zpracovat a vyhodnotit 3SAT problém pomocí jedné z pokročilých heuristik.
    Nejdříve jsem si musel napsat parser, který správně kopíroval informace ze vstupního souboru, takto vytažená data jsem poté ukládal do vlastní struktury, kterou jsem vytvořil
    a která sloužila k pozdějšímu zpracování problému.
    Jako pokročilou heuristiku jsem zvolil genetický algoritmus (GA), implementoval sem jeho jednotlivé části jako například turnajový výběr, mutaci a křížení. Poté jsem sledoval chování algoritmu
    a implementoval jsem adaptační mechanizmy, elitarismus a hromadné vymírání, jejich aplikování sem docílil toho, že můj algoritmus začal podávat přesnější a jednotnější výsledky.
    V rámci měření sem se zaměřil i na to jak velikost populace a počet generací ovlivňuje přesnost a čas běhu programu, větší populace po delší dobu vývoje (větší počet generací) vykazují
    přesnější výsledky, ale také delší dobu běhu, což je něco, co sem očekával.
    Z naměřených výsledků je patrné, že algoritmus se umí rychle přiblížit výsledku, a v části případů i odhadne správný výsledek, jeho běh ovšem není vždy bezchybný a je zde prostor pro zlepšení.

\end{document}